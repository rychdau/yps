\documentclass{amsart}

\usepackage{enumerate}

\title{Solutions for Chapter 1: Basic Properties of Numbers}
\author{Ryan Dau}
\date{\today}

\begin{document}

\maketitle

\begin{enumerate}[1.]
  \item
  \begin{enumerate}[(i)]
    \item If $ax=a$ for some number $a \neq 0$, then $x=1$.
  \begin{proof}
    Suppose $ax=a$ for some number $a \neq 0$. Then there exists its additive inverse $a^{-1}$, so that we have $(a \cdot a^{-1}) \cdot x= a \cdot a^{-1}=1$;
    thus, $1 \cdot x=1$, so $x=1$.
  \end{proof}
\item $x^{2}-y^{2}=(x-y)(x+y)$
  \begin{proof}
    $(x-y)(x+y)=x(x-y)+y(x-y)=x^2-xy+xy-y^2=x^2-y^2$
  \end{proof}
\item If $x^2=y^2$, then $x=y$ or $x=-y$.
  \begin{proof}
    Suppose $x^2=y^2$. This implies that $x^2-y^2=0$. The left-hand side is equal by 1.(ii) to $(x-y)(x+y)=0$. Thus, $x-y=0$ or $x+y=0$, so $x=y$ or $x=-y$.
  \end{proof}
\item $x^{3}-y^{3}=(x-y)(x^2+xy+y^2)$
  \begin{proof} $(x-y)(x^2+xy+y^2)=x^{2}(x-y)+xy(x-y)+y^{2}(x-y)=x^3-x^{2}y+xy^{2}-x^{2}y-xy^{2}-y^{3}=x^{3}-y^{3}$
  \end{proof}
\item $x^{n}-y^{n}=(x-y)(x^{n-1}+x^{n-2}y+ \cdots +xy^{n-2}+y^{n-1})$
  \begin{proof}
    $(x-y)(x^{n-1}+x^{n-2}y+ \cdots +xy^{n-2}+y^{n-1})=$
    \[(x-y)\sum_{i=1}^{n}(x^{n-i}y^{i-1})\]
    \begin{align*}
      (x-y)\sum_{i=1}^{n}(x^{n-i}y^{i-1}) &=\sum_{i=1}^{n}(x^{(n+1)-i}y^{i-1})-\sum_{i=1}^{n}(x^{n-i}y^{i})\\
                                          &= x^{n}+\sum_{i=2}^{n}(x^{(n+1)-i}y^{i-1})-\sum_{i=1}^{n-1}(x^{n-i}y^{i})-y^{n}\\
                                          &= x^{n}+\sum_{i=2}^{n}(x^{(n+1)-i}y^{i-1})-\sum_{j=2}^{n}(x^{(n+1)-j}y^{j-1})-y^{n}\quad (j=i+1)\\
                                          &= x^{n}+\sum_{i=j=2}^{n}(x^{(n+1)-i}y^{i-1}-x^{(n+1)-j}y^{j-1})
                                           = x^{n}-y^{n}
     \end{align*}
   \end{proof}
   \pagebreak
 \item $x^{3}+y^{3}=(x+y)(x^2-xy+y^2)$
   \begin{proof}
     Given 1.(iv), let $y_{0}=-y$, where $y_0$ denotes the $y$-variable in that exercise, so that $(x-(-y)=x^{2}+x(-y)+(-y)^{2}$, which is equivalent to $(x+y)=(x^{2}-xy+y^{2})$.
   \end{proof}

   Note that, from 1.(v), with $y_0=-y$ defined as above, for an odd $n$ $$(x+y)\sum_{i=1}^{n}(x^{n-i}y^{i-1})=x^n-(-y)^{n}=x^n+y^n$$ because $n$ is odd. The terms for the expansion will alternate in sign from positive to negative according as whether the exponentiation of $y$ is even or odd.
 \end{enumerate}
\item Note that the ``proof'' assumes that $x=y$, which implies that $x-y=0$. Thus, the multiplicative inverse $(x-y)^{-1}=0^{-1}$ does not exist.
\item
  \begin{enumerate}[(i)]
  \item  $ \frac{a}{b}=\frac{ac}{bc}$, if $b,c \neq 0$.
    \begin{proof}
      Suppose $b,c \neq 0$. Then $\frac{a}{b}=ab^-1=(ac)(b^{-1}c^{-1})=(ac)(bc)^{-1}=\frac{ac}{bc}$.
    \end{proof}
  \item $\frac{a}{b}+\frac{c}{d}=\frac{ad+bc}{bd}$, if $b,d \neq 0$.
  \begin{proof}
    Suppose $b,d\neq0$. Then $\frac{ad+bc}{bd}=(ad+bc)(bd)^{-1}=(ad)(bd)^{-1}+(bc)(bd)^{-1}=(ad \cdot b^{-1}c^{-1}) + (bc \cdot b^{-1}d^{-1})=(ab^{-1})+(cd^{-1})=\frac{a}{b}+\frac{c}{d}$
  \end{proof}
\item $(ab)^{-1}=a^{-1}b^{-1}$, if $a,b \neq 0$.
  \begin{proof}
    First, note that $(ab)(ab)^-1=1$; second, note that $(ab)(a^{-1}b^{-1})=1$. Thus, $(ab)(ab)^{-1}=(ab)(a^{-1}b^{-1})$. Multiplying both sides of the equation by the multiplicative inverse $(ab)^{-1}$, we have $(ab)^{-1}=a^{-1}b^{-1}$.

    This demonstrates that these are two equivalent ways of denoting a unique multiplicative inverse for $ab$.
  \end{proof}
\item $\frac{a}{b} \cdot \frac{c}{d}=\frac{ac}{db}$, if $b,d \neq 0$.
  \begin{proof}
    Suppose $b, d \neq 0$. Then $\frac{a}{b} \cdot \frac{c}{d}=(ab^{-1}) \cdot (cd^{-1})=(ac)(db)^{-1}=\frac{ac}{db}$
  \end{proof}
\item $\left. \frac{a}{b} \right/ \frac{c}{d}=\frac{ad}{bc}$, if $b, c, d \neq 0$.
  \begin{proof}
    Suppose $b,c,d \neq 0$. Then $\left. \frac{a}{b} \right/ \frac{c}{d}=(ab^{-1})/(cd^{-1})=(ab^{-1})(cd^{-1})^{-1}=(ab^{-1})(c^{-1}d)=(ad)(bc)^{-1}=\frac{ad}{bc}$.
  \end{proof}
\item If $b,d \neq 0$, then $\frac{a}{b}=\frac{c}{d}$ if and only if $ad=bc$. Also determine when $\frac{a}{b}=\frac{b}{c}$.
  \begin{proof}
    
    Suppose in all cases that $b, d \neq 0$.
    
    First, suppose $\frac{a}{b}=\frac{c}{d}$. This implies that $ab^{-1}=cd^{-1}$. Multiplying both sides by $b, d$ we have $ad=bc$.

    Second, suppose $ad=bc$. Multiplying both sides by $b^{-1}, d^{-1}$, we $ab^{-1}=cd^{-1}$. This implies that $\frac{a}{b}=\frac{c}{d}$.
\end{proof}
Finally, consider the case when $\frac{a}{b}=\frac{b}{a}$. This assertion is equivalent to $a \cdot a = b \cdot b$, which means $a^{2}=b^{2}$. This in turn implies that $a=b$ or $a=-b$, which is the condition under which the original equality holds.
\end{enumerate}
\item
  \begin{enumerate}[(i)]
  \item $4-x<3-2x$

    $x<-1$
  \item $5-x^{2}<8$

    $x^{2}>-3$

    all $x$

  \item
    $5-x^{2}<-2$

    $x^{2}>7$

    $\lvert x \rvert>\sqrt{7}$

    $x>\sqrt{7}$ or $x<-\sqrt{7}$

  \item $(x-1)(x-3)>0$

    $x>3$ or $x<1$

  \item $x^{2}-2x+2>0$

    $x^{2}-2x+1+1>0$

    $(x-1)^{2}+1>0$

    $(x-1)^{2}>-1$, and $(x-1)^{2} \geq 0$, so all $x$.

    \item $x^{2}+x+1>2$

      $x^{2}+x+\frac{1}{4}>\frac{5}{4}$

      $(x+\frac{1}{2})^{2}>\frac{5}{4}$

      $x>\frac{-1+\sqrt{5}}{2}$ or $x<\frac{-1-\sqrt{5}}{2}$

    \item $x^{2}-x+10>16$

      $x^{2}-x-6=(x-3)(x+2)=0$

      $x>3$ or $x<-2$

    \item $x^{2}+x+1>0$

      $x^{2}+x+\frac{1}{4}-\frac{1}{4}+1>0$

      $(x+\frac{1}{2})^{2}>-\frac{3}{4}$

      all $x$

    \item $(x- \pi )(x+5)(x-3)>0$

      $x>\pi$ or $-5<x<3$

    \item $(x-\sqrt[3]{2})(x-\sqrt{2})>0$

      $x>\sqrt{2}$ or $x<\sqrt[3]{2}$.

    \item $2^{x}<8$

      $x<3$
    \item $x+3^{x}<4$

      $x<1$

    \item $\frac{1}{x}+\frac{1}{1-x}>0$

      $\frac{1}{x}+\frac{1}{1-x}=\frac{1}{x(1-x)}>0$

      $0<x<1$

    \item $\frac{x-1}{x+1}>0$

      $x<-1$ or $x>1$
    \end{enumerate}

  \item
    \begin{enumerate}[(i)]
    \item If $a<b$ and $c<d$, then $a+c<b+d$.
      \begin{proof}
        Suppose $a<b$ and $c<d$. Then $b-a>0$ and $d-c>0$, so $b-a$ and $d-c$ are in $P$. But then $(b-a)+(d-c)=(b+d)-a-c=(b+d)-(a+c)$ is in $P$. Thus, $a+c<b+d$.
      \end{proof}
    \item If $a<b$, then $-b<-a$.
      \begin{proof}
        Suppose $a<b$. Then $b-a>0$ so b-a=-$(-b)-(-(-a))=(-a)-(-b)$ is in $P$. Thus, $-b<-a$.
      \end{proof}

    \item If $a<b$ and $c>d$, then $a-c<b-d$.
\begin{proof}
      Suppose $a<b$ and $ c>d$. Then $c>d$ implies $-d<-c$, which implies in conjunction with our hypothesis that $a+(-c)=a-c<b-d=b+(-d)$.
    \end{proof}
  \item If $a<b$ and $c>0$, then $ac<bc$.
    \begin{proof}
      Suppose $a<b$ and $c>0$. This implies that $b-a, c(=c-0)$ are in $P$, which means that $c(b-a)=bc-ac$ is in $P$. Thus, $ac<bc$.
    \end{proof}
  \item If $a<b$ and $c<0$, then $ac>bc$.
    \begin{proof}
      Suppose $a<b$ and $c<0$. Then $0<-c$, so $a(-c)<b(-c)$, which implies that $ac>bc$.
    \end{proof}
  \item If $a>1$, then $a^{2}>a$.
    \begin{proof}
      Suppose $a>1$. Note that $1>0$ and thus $a>0$. $a>1$ implies that $a-1, a$ are in $P$. Thus, $a^{2}-a$ is in $P$, which means that $a^{2}>a$.
    \end{proof}
  \item If $0<a<1$, then $a^{2}<a$.
    \begin{proof}
      Suppose $0<a<1$. Then $1-a, a$ are in $P$, so $a-a^{2}$ is in P, which means that $a>a^{2}$.
    \end{proof}
  \item If $0 \leq a<b$ and $0 \leq c<d$, then $ac<bd$.
    \begin{proof}
      If $a=0$ or $c=0$ the proof is trivial. Thus, suppose $a,b \neq 0$. Then $0<a<b$ and $0<c<d$; in particular, $c>0$, so $ac<bc$. Similarly, $b>0$, so $bc<bd$. Thus, $ac<bc<bd$, so $ac<bd$.
    \end{proof}
  \item If $0 \leq a<b$, then $a^{2}<b^{2}$.
    \begin{proof}
      Suppose $0 \leq a<b$. Then we have immediately from 5.(vii) that $a^{2}<b^{2}$, letting $c=a$ and $d=a$.
    \end{proof}
  \item If $a, b \geq 0$ and $a^{2}<b^{2}$, then $a<b$.
    \begin{proof}
    Suppose $a, b \geq 0$ and $a^{2}<b^{2}$. Suppose $a \not<b$; this is equivalent to $a \geq b$.

    First, take the case $a=b$. This implies that $a^{2}=b^{2}$, contradicting our hypothesis.

    Second, take the case $a>b$. This implies, in cojunction with 5.(ix), that $a^{2}>b^{2}$, contradicting our hypothesis.

    Thus, it must be that $a<b$.\end{proof}
  \end{enumerate}
\item
  \begin{enumerate}[(a)]
  \item If $0 \leq x<y$, then $x^{n}<y^{n}$ for $n=1,2,3, \ldots$.
    \begin{proof}
      We will use induction on $n$.

      First, suppose that $0 \leq x<y$. This immediately implies the base case $n=1$, $x^{1}=x<y^{1}=y$.

      Next, assume the inductive hypothesis: $x^{n}<y^{n}$. We must show that $x^{n+1}<y^{n+1}$. First, we have that $x \cdot x^{n}=x^{n+1}<x \cdot y^{n}$. Second, since $y>0$ and thus $y^{n}>0$, we have that $x cdot y^{n}<y^{n} \cdot y=y^{n+1}$. Thus, $x^{n+1}<x \cdot y^{n}<y^{n+1}$, so $x^{n+1}<y^{n+1}$. Thus, the inequality holds for all $n$.
    \end{proof}
  \item If $x<y$ and $n$ is odd, then $x^{n}<y^{n}$.
    \begin{proof}
      Let $n$ be odd, and suppose $x<y$.

      First, suppose $x, y \geq 0$. This implies, in conjunction with 6.(a), that $x^{n}<y^{n}$.

      Second, suppose $x,y \leq 0$. This implies by the trichotomy law that $-x, -y \geq 0$, so $-y<-x$, which implies in conjunction with 6.(a) that $(-y)^{n}<(-x)^{n}$. Since $n$ is odd, we have $-y^{n}<-x^{n}$, which is equivalent to $x^{n}<y^{n}$.

      Finally, suppose $x \leq 0 < y$. Since $n$ is odd and $x$ is negative (the case is trivial when $x=0$), $x^{n}$ will be negative for every odd $n$. Similarly, $y$ will be positive for any odd $n$. Thus, $x^{n} \leq 0 <y^{n}$, so $x^{n}<y^{n}$.

      Thus, in all cases, $x^{n}<y^{n}$.
    \end{proof}
  \item If $x^{n}=y^{n}$ and $n$ is odd, then $x=y$.
\begin{proof}
    Suppose $x^{n}=y^{n}$ and $n$ is odd. Let us suppose that $x \neq y$, i.e., that $x>y$ or $x<y$.

    First, take $x>y$. This implies that $x^{n}>y^{n}$, contradicting our hypothesis.

    Second, take $x<y$. This implies that $x^{n}<y^{n}$, contradicting our hypothesis.

    Thus, $x=y$.
  \end{proof}
\item If $x^{n}=y^{n}$ and $n$ is even, then $x=y$ or $x=-y$.
  \begin{proof}
    Suppose in all cases that $x^{n}=y^{n}$

    First, suppose $x, y \geq 0$.Suppose further that $x \neq y$. Without loss of generality, suppose $x>y$. This implies that $x^{n}>y^{n}$, contradicting our hypothesis (and likewise for $x<y$). Thus, when $x, y \geq 0$, $x=y$.

    Second, suppose $x, y \leq 0$. Suppose further that $x \neq y$. This implies that $-x, -y \geq 0$. $x \neq y$. Without loss of generality, suppose $-x>-y$. This implies that $(-x)^n>(-y)^{n}$, and since $n$ is even, $x^{n}>y^{n}$, contradicting our hypothesis. Thus, when $x, y \leq 0$, $x=y$.

    Third, suppose $x$ and $y$ are of opposite signs. Suppose $x \neq y$. Without loss of generality, suppose $x<0<y$. Thus, $0<-x,y$. Without loss of generality again, suppose $-x<y$. But then $(-x)^{n}=x^{n}<y^{n}$, contradicting our hypothesis. Thus, when $x$ and $y$ are of opposite signs, $x=-y$.
  \end{proof}
  \end{enumerate}

\item If $0<a<b$, then $a<\sqrt{ab}<\frac{a+b}{2}<b$.
  \begin{proof}
    Suppose $0<a<b$.

    Noting that $b>0$, we have that $a+b<b+b=2b$, which implies that $\frac{a+b}{2}<b$.

    Noting that $a>0$, we have that $a \cdot a = a \cdot b$, which implies that $a<\sqrt{ab}$.

    Finally, noting that $(a-b)^{2}=(a-2ab+b^{2})>0
    $, we have that $a+2ab+b^{2}>4ab$, which implies that $(a+b)^{2}>4ab$, so $a+b>2\sqrt{ab}$. Thus, $\frac{a+b}{2}\sqrt{ab}$.
  \end{proof}
\item Suppose that P10--P12 are replaced by
  \begin{enumerate}
\item[(P$'$10)] For any numbers $a$ and $b$ one, and only one, of
  the following holds:
  \begin{enumerate}
  \item[i.] $a=b$
  \item[ii.] $a<b$
  \item[iii.] $b<a$.
  \end{enumerate}
\item[(P$'$11)] For any numbers $a$, $b$, and $c$, if $a<b$ and $b<c$, then
  $a<c$.
\item[(P$'$12)] For any numbers $a$, $b$, and $c$, if $a<b$, then $a+c<b+c$.
\item[(P$'$13)] For any numbers $a$, $b$, and $c$, if $a<b$ and $0<c$,
  then $ac<bc$.
\end{enumerate}

Show that P10--P12 can then be deduced as theorems.

\begin{proof}
  First, by two applications of (P$'$11) we have, for numbers $a$,
  $b$, $c$, and $d$, with $a<b$ and $c<d$, $a+c<b+c$ and $b+c<b+d$, which by
  (P$'$11) implies $a+c<b+d$. Note in particular that when $a=0$ and $b=0$,
  we have that $0<b$ and $0<d$ implies that $0<b+d$, which is of course
  (P11).

  Second, suppose (P10) were not to hold: this would imply that more than
  one of $a=0$, $a$ is in the collection $P$, and $-a$ is in the collection $P$
  would hold. Thus, using the proof above, let $a<0$ and $-a<0$, noting that
  both can hold under the negation of (P10): we have $-a$ in the collection
  $P$ and $a$ in the collection $P$. Then the above proof results in $0=-a+a<0$,
  or $0<0$. But this contradicts the postulate (P$'$10): note that we have
  $0=0$ by the properties of equality, and $0<0$, but only one of the three
  conditions can hold. Thus, our assumption that (P10) does not hold leads
  to an absurdity, and we can conclude that (P10) does hold.

  Third, suppose we have numbers $a$, $b$, $c$ such that $a<b$ and $c<d$ with
  $a>0$ and $c>0$. Then, by two applications of (P$'$13) we have $ac<bc$ and
  $bc<bd$, which by (P$'$11) implies that $ac<bd$. Note that, since $a,c>0$ we have that $b,d>0$. Similarly, since $0<a$ an $0<c$, we have that $b \cdot 0=0<ac$. Thus, $0<ac$, and $ac<bd$, so $0<bd$. Put differently, we have shown that when $0<b,d$, then $0<b \cdot d$.

We can conclude that from assumptions concerning the behavior of inequalities over numbers, we can deduce the trichotomy law and closure under addition and multiplication as theorems.
  
\end{proof}
\item
  \begin{enumerate}[(i)]
  \item $\lvert \sqrt{2}+\sqrt{3}-\sqrt{5}+\sqrt{7} \rvert = \sqrt{2}+\sqrt{3}-\sqrt{5}+\sqrt{7}$
  \item $\lvert (\lvert a+b \rvert-\lvert a \rvert - \lvert b \rvert) \rvert=\lvert a+b\rvert + \lvert a \rvert + \lvert b \rvert$.
  \item $\lvert(\lvert a+b \rvert + \lvert c \rvert - \lvert a+b+c \rvert)\rvert$
  \item $\lvert x^{2}-2xy+7^{2}\rvert= \lvert (x-y)^{2} \rvert = (x-y)^{2} = x^{2}-2xy-y^{2}$
    \item $\lvert (\lvert \sqrt{2}+ \sqrt{3} \rvert - \lvert \sqrt{5}-\sqrt{7}\rvert)\rvert= \lvert \sqrt{2}+\sqrt{3}+\sqrt{5}-\sqrt{7} \rvert= \sqrt{2}+\sqrt{3}+\sqrt{5}-\sqrt{7}$
    \end{enumerate}

  \item
    \begin{enumerate}[(i)]
  \item $\lvert a+b \rvert - \lvert b \rvert$
    \begin{align*}
      a&, \quad \text{if}\quad a+b\geq 0, b\geq0 \\
      a+2b&, \quad \text{if}\quad a+b \geq 0, b \leq 0\\
      -a&, \quad \text{if}\quad a+b\leq0, b\geq0\\
      2b-a&, \quad \text{if}\quad a+b \leq 0, b\leq0
    \end{align*}
  \item $\lvert ( \lvert x \rvert-1)\rvert $
    \begin{align*}
      x-1&, \quad \text{if}\quad x\geq 1\\
      -x-1&, \quad \text{if}\quad x\leq -1\\
      1-x&, \quad \text{if}\quad 0\leq x \leq1\\
      1+x&, \quad \text{if}\quad -1 \leq x \leq0
    \end{align*}
    
  \item $\lvert x \rvert-\lvert x^{2} \rvert=\lvert x \rvert -x^{2}$
    \begin{align*}
      x-x{^2}&, \quad \text{if}\quad x\geq0\\
      -x-x^{2}&, \quad \text{if}\quad x \leq0
    \end{align*}
  \item $a-\lvert (a - \lvert a \rvert)\rvert$
    \begin{align*}
      a&, \quad \text{if}\quad a\geq0\\
      3a&, \quad \text{if}\quad a\leq0
    \end{align*}
  \end{enumerate}
\item
  \begin{enumerate}[(i)]
  \item $\lvert x-3 \rvert=8$
    
    $x=11$ or $x=-5$
  \item $\lvert x-3 \rvert <8$
    
    $-5<x<11$
  \item $\lvert x+4 \rvert <2$
    
    $-6<x<-2$
  \item $\lvert x-1 \rvert + \lvert x-2 \rvert>1$
    
    $x<1$ or $x>2$
  \item $\lvert x-1 \rvert + \lvert x+1 \rvert<2$
    
    $-1<x<1$
  \item $\lvert x-1 \rvert + \lvert x+1 \rvert <1$
    
    no $x$
  \item $\lvert x-1 \rvert \cdot \lvert x+1 \rvert = 0$
    
    $x=1$ or $x=-1$
  \item $\lvert x-1 \rvert \cdot \lvert x+2 \rvert=3$
    
    $x=\frac{-1-\sqrt{21}}{2}$
    \end{enumerate}
  \end{enumerate}
\end{document}